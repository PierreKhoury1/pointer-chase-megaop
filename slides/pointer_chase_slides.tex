\documentclass[aspectratio=169]{beamer}

\usetheme{Madrid}
\usecolortheme{dolphin}
\usepackage{graphicx}
\usepackage{amsmath}
\usepackage{booktabs}
\usepackage{hyperref}

\graphicspath{{../build/graphs/}{../build/graphs/combined_graphs/}{../build/graphs/aot_vs_hotspot_graphs/}{../atomic_chase/graphs_callgrind/}}

\title{Pointer Chasing as a Semantic Mega-Instruction}
\subtitle{AOT + HotSpot Evidence and the Atomic Chase Prototype}
\author{Pointer-Chase Study}
\date{\today}

\begin{document}

\begin{frame}
  \titlepage
\end{frame}

\begin{frame}{Agenda}
  \begin{enumerate}
    \item Motivation and hypothesis
    \item AOT evidence (GCC/Clang)
    \item HotSpot evidence (JVM)
    \item Combined view
    \item Atomic chase prototype
    \item Status and next steps
  \end{enumerate}
\end{frame}

\begin{frame}{Motivation}
  Pointer chasing forces a serial dependency chain:
  \[
    p_{i+1} = *\left(p_i + \Delta_i\right), \quad i = 0..(d-1)
  \]
  This typically expands to:
  \[
    \mathcal{O}(d)\ \text{loads} + \mathcal{O}(d)\ \text{guards} + \mathcal{O}(d)\ \text{control-flow}
  \]
  Goal: justify representing the entire chase as a single semantic unit.
\end{frame}

\begin{frame}{Why This Approach?}
  \begin{itemize}
    \item We need structural evidence, not only timing results.
    \item AOT shows what compilers emit; JIT shows guards and deopts.
    \item If both retain \(\mathcal{O}(d)\) structure, a single semantic unit is justified.
  \end{itemize}
\end{frame}

\begin{frame}{AOT Experimental Matrix}
  \textbf{Depths:} 2, 3, 5, 8 \\
  \textbf{Variants:}
  \begin{itemize}
    \item Plain dereference: \texttt{**p, ***p, ...}
    \item Constant offsets: \texttt{*(p + c1) -> *(... + c2) -> ...}
    \item Dynamic offsets: offsets passed as arguments
    \item Explicit checks: per-level guard calls
  \end{itemize}
  \vspace{0.3em}
  \textbf{Metrics:} loads, branches, basic blocks, total instructions.
\end{frame}

\begin{frame}{AOT Snapshot (GCC -O2)}
  \centering
  \small
  \begin{tabular}{lrrrr}
\toprule
Pattern & Loads d2 & Loads d8 & Branches d2 & Branches d8 \\
\midrule
plain & 2 & 8 & 0 & 0 \\
const & 2 & 8 & 0 & 0 \\
dyn & 2 & 11 & 0 & 0 \\
checks & 2 & 11 & 2 & 8 \\
\bottomrule
\end{tabular}

  \vspace{0.5em}
  \begin{itemize}
    \item Loads scale with depth for all patterns.
    \item Branches appear primarily with explicit checks.
  \end{itemize}
\end{frame}

\begin{frame}{AOT Results: Combined (All Compilers)}
  \begin{columns}
    \column{0.5\textwidth}
      \includegraphics[width=\linewidth,height=0.32\textheight,keepaspectratio]{combined_loads.png}
      \vspace{0.3em}
      \includegraphics[width=\linewidth,height=0.32\textheight,keepaspectratio]{combined_branches.png}
    \column{0.5\textwidth}
      \includegraphics[width=\linewidth,height=0.32\textheight,keepaspectratio]{combined_basic_blocks.png}
      \vspace{0.3em}
      \includegraphics[width=\linewidth,height=0.32\textheight,keepaspectratio]{combined_instructions.png}
  \end{columns}
\end{frame}

\begin{frame}{AOT Takeaways}
  \begin{itemize}
    \item Loads scale linearly with depth across compilers and optimizations.
    \item Explicit checks induce \(\mathcal{O}(d)\) branches and basic blocks.
    \item Optimization shrinks constants, but \(\mathcal{O}(d)\) scaling remains.
  \end{itemize}
\end{frame}

\begin{frame}{HotSpot (JVM) Measurement}
  We extended the same matrix to HotSpot and extracted:
  \[
    \text{deref ops},\; \text{guards (uncommon\_trap)},\; \text{deopts},\; \text{IR proxy counts}
  \]
  Note: Product JVMs do not expose C2 IR nodes. We use bytecode parse
  statistics as proxies and log all \texttt{uncommon\_trap} entries.
\end{frame}

\begin{frame}{HotSpot Snapshot (Proxies)}
  \centering
  \small
  \begin{tabular}{lrrrr}
\toprule
Pattern & Deref d2 & Deref d8 & Guards d2 & Guards d8 \\
\midrule
plain & 2 & 8 & 2 & 8 \\
const & 4 & 16 & 6 & 24 \\
dyn & 2 & 8 & 6 & 24 \\
checks & 2 & 8 & 4 & 16 \\
\bottomrule
\end{tabular}

  \vspace{0.5em}
  \begin{itemize}
    \item Deref ops scale with depth.
    \item Guards scale roughly with depth, mirroring AOT.
  \end{itemize}
\end{frame}

\begin{frame}{AOT vs HotSpot: Loads / Deref Ops}
  \begin{center}
    \includegraphics[width=0.9\linewidth,height=0.7\textheight,keepaspectratio]{aot_vs_hotspot_loads.png}
  \end{center}
\end{frame}

\begin{frame}{AOT vs HotSpot: Branches / Guards}
  \begin{center}
    \includegraphics[width=0.9\linewidth,height=0.7\textheight,keepaspectratio]{aot_vs_hotspot_branches.png}
  \end{center}
\end{frame}

\begin{frame}{AOT vs HotSpot: Instructions / IR Proxy}
  \begin{center}
    \includegraphics[width=0.9\linewidth,height=0.7\textheight,keepaspectratio]{aot_vs_hotspot_instructions.png}
  \end{center}
\end{frame}

\begin{frame}{HotSpot Takeaways}
  \begin{itemize}
    \item Deref operations scale linearly with depth.
    \item Guards and deopt points also scale roughly \(\mathcal{O}(d)\).
    \item This mirrors the AOT structure and motivates a single semantic unit.
  \end{itemize}
\end{frame}

\begin{frame}[fragile]{Atomic Chase: Semantic Mega-Instruction Model}
  \textbf{Naive (current practice):}
  \begin{verbatim}
for (i = 0; i < depth; i++) {
    p = p->next;
}
return p->value;
  \end{verbatim}
  \textbf{Atomic (mega-instruction model):}
  \begin{verbatim}
if (!p || !p->next || ... ) return -1;
Node *q = p->next->next->...;
return q->value;
  \end{verbatim}
  Checks are upfront; dataflow is straight-line with a single return.
\end{frame}

\begin{frame}[fragile]{Code Example: Naive Chase (C)}
  \begin{verbatim}
int chase_naive(Node *p, int depth) {
    for (int i = 0; i < depth; i++) {
        p = p->next;
    }
    return p->value;
}
  \end{verbatim}
\end{frame}

\begin{frame}[fragile]{Code Example: Atomic Chase (C)}
  \begin{verbatim}
int chase_atomic(Node *p, int depth) {
    if (!p || !p->next || !p->next->next) return -1;
    Node *q = p->next->next;
    return q->value;
}
  \end{verbatim}
  (Actual implementation is unrolled for depths 2,3,5,8.)
\end{frame}

\begin{frame}[fragile]{Code Example: Atomic Chase (Switch Excerpt)}
  \begin{verbatim}
switch (depth) {
  case 5:
    if (p && p->next && p->next->next &&
        p->next->next->next &&
        p->next->next->next->next &&
        p->next->next->next->next->next) {
      result = p->next->next->next->next->next->value;
    }
    break;
}
  \end{verbatim}
  This models upfront checks + straight-line dataflow per depth.
\end{frame}

\begin{frame}[fragile]{Code Example: Benchmark Harness (C)}
  \begin{verbatim}
static volatile int sink;
for (int i = 0; i < ITERS; i++) {
    acc ^= chase_naive(head, depth);
}
for (int i = 0; i < ITERS; i++) {
    acc ^= chase_atomic(head, depth);
}
sink = acc;
  \end{verbatim}
\end{frame}

\begin{frame}{Why Atomic Chase Now?}
  \begin{itemize}
    \item The AOT/JIT evidence shows persistent \(\mathcal{O}(d)\) structure.
    \item Atomic chase isolates the effect of control-flow collapse.
    \item We avoid changing algorithms to keep comparisons honest.
  \end{itemize}
\end{frame}

\begin{frame}{Atomic Chase Benchmark Harness}
  \begin{itemize}
    \item Depths: 2, 3, 5, 8
    \item Tight loops (\(\ge 10^7\) iters), identical inputs
    \item Metrics via \texttt{perf stat}:
      \(\text{instructions, cycles, branches, branch-misses}\)
  \end{itemize}
  Note: hardware counters are unavailable in WSL; perf runs must be done
  on a native Linux machine.
\end{frame}

\begin{frame}{Callgrind (WSL-safe) Results}
  \begin{columns}
    \column{0.5\textwidth}
      \includegraphics[width=\linewidth,height=0.33\textheight,keepaspectratio]{callgrind_Ir.png}
      \vspace{0.3em}
      \includegraphics[width=\linewidth,height=0.33\textheight,keepaspectratio]{callgrind_Bc.png}
    \column{0.5\textwidth}
      \begin{itemize}
        \item Ir = dynamic instruction refs (simulated)
        \item Bc = conditional branches executed
        \item Atomic has \emph{more} branches due to explicit null checks
      \end{itemize}
  \end{columns}
\end{frame}

\begin{frame}{Callgrind: Branch Mispredicts}
  \begin{center}
    \includegraphics[width=0.85\linewidth,height=0.7\textheight,keepaspectratio]{callgrind_Bcm.png}
  \end{center}
  Mispredicts are low and nearly flat; branches are predictable in this synthetic setup.
\end{frame}

\begin{frame}{Current Direction}
  \begin{enumerate}
    \item Run perf on native Linux for atomic vs naive.
    \item Prioritize branches and branch-misses as the primary signal.
    \item If the structure collapses, formalize the semantic contract.
  \end{enumerate}
\end{frame}

\begin{frame}{Status}
  \begin{itemize}
    \item \textbf{Done:} AOT tables + graphs (GCC/Clang, -O0/-O2)
    \item \textbf{Done:} HotSpot tables + graphs (bytecode/guard proxies)
    \item \textbf{Done:} Atomic chase code + benchmark harness
    \item \textbf{Done:} Callgrind (WSL-safe) branch/instruction estimates
    \item \textbf{Pending:} Native Linux perf run + hardware counters
  \end{itemize}
\end{frame}

\begin{frame}{Next Steps}
  \begin{enumerate}
    \item Run perf on native Linux (collect \texttt{instructions}, \texttt{branches}, \texttt{branch-misses})
    \item Produce atomic-vs-naive graphs and summary table
    \item (Optional) Use fastdebug JVM to extract real C2 IR node counts
  \end{enumerate}
\end{frame}

\begin{frame}{Conclusion}
  The data show a persistent \(\mathcal{O}(d)\) structural cost in both
  AOT and JIT pipelines. The atomic chase model provides a concrete
  semantic target for collapsing this structure into a single unit.
\end{frame}

\end{document}
